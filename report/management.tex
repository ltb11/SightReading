\section{Project Management}
\subsection{Supervisorial input}
We arranged to have weekly meetings with our supervisor William Knottenbelt to keep him updated on the progress of our project, whilst also ensuring we felt we had a target to work to. We had an aggressive implementation schedule at the beginning of the project, and  This worked well for the first few weeks but as the term progressed and we had more time commitments we found it more difficult to keep to the schedule.
\subsection{Meetings}
We held meetings every Tuesday since the start of the project.The aims of the meeting were as follows:
\begin{itemize}
	\item Discuss what went well/badly in the previous week. How can we improve/learn from our mistakes?
	\item Make amendments to our plan in order to make sure we haven�t overestimated/underestimated the amount of work that we can do over the coming weeks.
	\item Work out how to best split up the following week�s tasks.
\end{itemize}
As well as this, larger weekly meeting to give us direction for the week, we also had as many smaller daily meetings as our timetables will allow. This allows us to spread the knowledge among group members. It also means that the sub groups we have split into can stay in contact which helps to keep the project integrated, cut down on duplication of work and allow other people on the group who are not working directly on that problem to give constructive feedback.

We have also been using pair programming to spread knowledge around the team so that all members have a good understanding of how the product works. Again, this helps to keep us flexible and agile.

\subsection{Trello}
The project management tool Trello was useful for us because of the highly itemized nature of our project.
