\section{Background}
    \subsection{Musical Notation}
        In order properly to understand the task of reading music, one must understand a few basic terms.

        \subsubsection{Pitch}
            The pitch of a note is referred to by a one of the first 7 letters of the alphabet, i.e. \{C,D,E,F,G,A,B\}.
            In scientific pitch notation, a number from 0 - 8 is added after the letter to indicate which octave the note is in.
            The lowest note possible in this notation, C0, represents a sound with frequency 16.35Hz, and the heighest possible, B8, has a frequency 7902.13Hz.
            A4 (frequency 440Hz) is considered an octave lower than A5 (frequency 880Hz), and an increase in octave represents a doubling of frequency in the sound produced.
        \subsubsection{Duration}
            In addition to pitch, notes also encode information about their duration, or how long the sound should be played. 
            Rests indicate an absence of sound. Both these durations are encoded as in \ref{table:notes}

            \begin{table}[h]
                \centering
                \begin{tabular}{| c | c | c |}
                    \hline
                    Name & Note & Rest \\ \hline
                    Minim& 
                            \includegraphics[width=20mm]{./assets/half.png}
                            &
                            \includegraphics[width=20mm]{./assets/halfrest.png} \\ \hline
                    Quaver&
                            \includegraphics[width=20mm]{./assets/4er.png}
                            &
                            \includegraphics[width=20mm]{./assets/4errest.png} \\ \hline
                    Semiquaver&
                            \includegraphics[width=20mm]{./assets/8th.png}
                            &
                            \includegraphics[width=20mm]{./assets/8threst.png} \\ \hline
                    Crotchet&
                            \includegraphics[width=20mm]{./assets/16th.png}
                            &
                            \includegraphics[width=20mm]{./assets/16threst.png} \\ \hline
                \end{tabular}
                \caption{Note and Rest Durations}
                \label{table:notes}
            \end{table}


               
        \subsubsection{Stave}
            A stave comprises five horizontal lines and four spaces. A note can sit either in a space or on a line, and the height of the note dictates its pitch.
            \begin{figure}[ht!]
            \centering
            \includegraphics[width=90mm]{./assets/staff.png}
            \caption{The Stave}
            \label{stave}
            \end{figure}
        \subsubsection{Clef}


    \subsection{Computer Vision and OMR}
