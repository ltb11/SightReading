\section{Background}
    \subsection{Musical Notation}
        In order properly to understand the task of reading music, one must understand a few basic terms.

        \subsubsection{Pitch}
            The pitch of a note is referred to by a one of the first 7 letters of the alphabet, i.e. \{C,E,F,G,A,B\}.
            In addition to this, in scientific pitch notation, a number from 0 - 8 is added after the letter to indicate which octave the note is in.
            The lowest note possible in this notation, C0, represents a sound with frequency 16.35Hz, and the heighest possible, B8, has a frequency 7902.13Hz.
            A4 (frequency 440Hz) is considered an octave lower than A5 (frequency 880Hz), and an increase in octave represents a doubling of frequency in the sound produced.
        \subsubsection{Note}
               
        \subsubsection{Stave}
            A stave comprises five horizontal lines and four spaces. A note can sit either in a space or on a line, and the height of the note dictates its pitch.
            \begin{figure}[ht!]
            \centering
            \includegraphics[width=90mm]{./assets/staff.png}
            \caption{The Stave}
            \label{stave}
            \end{figure}


    \subsection{Computer Vision and OMR}
