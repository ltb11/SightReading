\section{Evaluation}
    
    \subsection{Performance}
    We can breakdown the performance of our app into three major areas
        \begin{enumerate}
        \item{How accurately the detected music reflects the actual music}
        \item{The speed with which our app parses the sheet music}
        \end{enumerate}
        \subsubsection{Accuracy of Detection}
            We tested the accuracy of the music detection by parsing various different test images and checking by hand how many of the features were correctly detected. We considered trying to automate this process when we first started the project, however we quickly realised that solving this automation problem would take just as much if not more time than solving the task itself. As such it is a method that has stuck with us since the beginning.
            The results of our most recent tests are seen in Table \ref{table:testresults}
            \begin{table}[h!]
                \centering
                \begin{tabular}{ c | c | c }
                    Piece Name & Distinct Feature Types & Percentage of Features Correctly Detected \\
                    Baa Baa Black Sheep & 8 & 100\% \\
                    Indiana Jones & 10 & 90\%\\
                \end{tabular}
                \caption{Accuracy Test Results}
                \label{table:acctestresults}
            \end{table}
            
            All of the basic feature types, which is to say all the distinct feature types in "Baa Baa Black Sheep" are reliably detected in any piece we try to parse. However the more complex elements, such as the Rests in Indiana Jones are what cause problems for our detection system.
        \subsubsection{Speed of Detection}
        We measured the speed of detection by logging the times at which various tasks were started and finished, and then measuring the time between the two. Speed was very important to us, as a user can very quickly lose interest if the app produces no output after a long time. Because of these measurements, we determined at several stages in development that the algorithms we were using to detect features were often far too slow for our needs, and a faster (potentially less
        accurate) algorithm was necessary (as mentioned in our Implementation). Our approach to the trade off was this: As long as the detection and playback happened within 30 seconds, accuracy was the most important aspect. 
            \begin{table}[h!]
                \centering
                \begin{tabular}{ c | c | c | c }
                    Device & Android Version & Piece & Time \\
                    Galaxy S3 & 4.1 & Baa Baa Black Sheep & 5-10s \\
                    Motorola Xoom & 4.2.2 & Baa Baa Black Sheep & 10-20s
                \end{tabular}
                \caption{Speed Test Results}
                \label{table:speedtestresults}
            \end{table}
            Our speed performance varies from device to device, but in general we are wll within our 30s aim.
    Overall our app runs at a speed that is reasonable and with accuracy on basic sheet music that we are satisfied with.
    \subsection{Difficulties Encountered}
    We encountered several major difficulties when developing our app.
    \begin{enumerate}
        \item{The lack of documentation for OpenCV and in particular OpenCV4Android. For instance: we attempted to write our own methods to iterate over pixels in an image, and found that they ran very slowly compared to the native methods. It took us a long time to work out that this was because the internal methods were all running C++ code, whereas our implementations would only ever run Java code}
        \item{Revision control and }

    \end{enumerate}

